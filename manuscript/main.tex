\documentclass[12pt, titlepage]{article}

\usepackage[margin = 1in]{geometry}
\usepackage{amsmath}
\allowdisplaybreaks
\usepackage{authblk}
\usepackage{setspace}
\usepackage{natbib}
\usepackage{hyperref}\hypersetup{colorlinks=true, citecolor=blue}


\usepackage[]{lineno}
\linenumbers*[1]
%% patches to make lineno work better with amsmath
\newcommand*\patchAmsMathEnvironmentForLineno[1]{%
	\expandafter\let\csname old#1\expandafter\endcsname\csname
	#1\endcsname
	\expandafter\let\csname oldend#1\expandafter\endcsname\csname
	end#1\endcsname
	\renewenvironment{#1}%
	{\linenomath\csname old#1\endcsname}%
	{\csname oldend#1\endcsname\endlinenomath}}%
\newcommand*\patchBothAmsMathEnvironmentsForLineno[1]{%
	\patchAmsMathEnvironmentForLineno{#1}%
	\patchAmsMathEnvironmentForLineno{#1*}}%
\AtBeginDocument{%
	\patchBothAmsMathEnvironmentsForLineno{equation}%
	\patchBothAmsMathEnvironmentsForLineno{align}%
	\patchBothAmsMathEnvironmentsForLineno{flalign}%
	\patchBothAmsMathEnvironmentsForLineno{alignat}%
	\patchBothAmsMathEnvironmentsForLineno{gather}%
	\patchBothAmsMathEnvironmentsForLineno{multline}%
}

% control floats
\renewcommand\floatpagefraction{.9}
\renewcommand\topfraction{.9}
\renewcommand\bottomfraction{.9}
\renewcommand\textfraction{.2}
\setcounter{totalnumber}{50}
\setcounter{topnumber}{50}
\setcounter{bottomnumber}{50}


\newcommand{\jy}[1]{\textcolor{orange}{JY: (#1)}}
\newcommand{\yh}[1]{\textcolor{purple}{YH: (#1)}}
\newcommand{\yx}[1]{\textcolor{cyan}{YX: (#1)}}
\let\proglang=\textsf
%% \newcommand{\pkg}[1]{{\fontseries{m}\selectfont #1}}
%% \newcommand\code[2][black]{\textcolor{#1}{\texttt{#2}}}



\title{Principles for Open Data Curation: A Case Study with the New
  York City 311 Service Request Data}


\author[1]{David Tussey}
\author[2]{Jun Yan}
\affil[1]{Former Executive Director, NYC DoITT}
\affil[2]{Department of Statistics, University of Connecticut}


\begin{document}
\maketitle


\begin{abstract}
  Open data is important ...
  Quality control is important ...
  We use New York City 311 requests data  ...
  We suggest some open data quality control protocols ...


\bigskip
  
\noindent
{\it Keywords:}
consistency check;    
data release protocol;
quality control
\end{abstract}

\doublespacing

\section{Introduction} \label{sec:intro}

In the early 21st century, the open data movement began to take shape,
driven by the fundamental belief that freely accessible data can
transform societies. This movement champions the principles of
transparency, innovation, and public engagement. A landmark in this
journey was the launch of the United States'
\href{https://www.data.gov}{Data.gov} portal in 2009, a pioneering
platform in making government data widely accessible. Shortly after,
the European Union followed suit, unveiling its
\href{https://data.europa.eu/euodp}{Open Data Portal} in 2012, further
cementing the movement's global reach. These initiatives represent
significant strides in democratizing data, breaking barriers that once
kept valuable information in silos. Furthermore, the World Bank's Open
Data initiative, initiated in 2010, stands out as a comprehensive
repository for global development data, available at
\href{https://data.worldbank.org}{World Bank Open Data}. The
collective impact of these platforms is profound, extending beyond
mere data sharing to fostering a culture of openness that benefits
individuals, communities, and economies worldwide
\citep{janssen2012benefits, barns2016mine, wang2016adoption}.


New York City (NYC) has emerged as a forerunner in the open data
movement, marked by the enactment of the Open Data Law in 2012
\citep{zuiderwijk2014open}. This landmark legislation led to the
creation of the \href{https://opendata.cityofnewyork.us}{NYC Open Data
  portal}, which today hosts an impressive array of 3,511 datasets
from 98 different city agencies. This resource has become invaluable
for researchers in various fields. In health, datasets have enabled
significant studies on facets of health and healthcare delivery
\citep{cantor2018facets, shankar2021data}. In the realm of urban
development, data has been instrumental in advancing smart city
initiatives \citep{neves2020impacts}. Additionally, transportation
research has benefited greatly from this wealth of data, aiding in the
understanding of urban mobility and infrastructure
\citep{gerte2019understanding}. NYC's Open Data initiative not only
exemplifies commitment to transparency and public engagement but also
illustrates how open data can be a powerful tool in addressing complex
urban challenges.


Data curation is fundamental in the open data ecosystem, ensuring the
utility and reliability of datasets for diverse applications. Among
the earliest discussions, \citet{witt2009constructing} focus on the
development of data curation profiles tailored to specific contexts,
setting a precedent for targeted data management
strategies. Addressing broader challenges in data sharing and
management, \citet{borgman2012conundrum} highlights the complexities
of research data distribution, emphasizing the need for robust
strategies. This is complemented by the work of \citet{hart2016ten},
who outline essential principles for effective data management,
particularly emphasizing the importance of meticulous curation
practices. In the realm of collaborative data management,
\citet{beheshti2019datasynapse} underscore the significance of
cooperative environments for managing and sharing social data
effectively. This aspect of data curation gains further relevance in
the research by \citet{mclure2014data}, which delves into the specific
practices and needs within data curation communities. The practical
implications of data curation are vividly illustrated in the context
of public health and global challenges. \citet{cantor2018facets}
demonstrate the utility of curated open data in evaluating community
health determinants. Furthermore, the COVID-19 pandemic serves as a
real-world example, with \citet{shankar2021data} observing the
critical role of collective data curation efforts in managing and
responding to the crisis. Collectively, these studies not only
highlight the multifaceted nature of data curation but also emphasize
its indispensable role in enhancing the applicability and value of
open data across various domains.


The contributions of this paper are twofold. Firstly, we delve into
the specifics of data curation challenges using the NYC 311 Service
Request Data as a case study. This renowned open dataset serves as a
prime example for examining key issues in data curation, including
data validity, consistency, and curation efficiency. We illustrate
these points with live examples drawn from our processing of the 311
data. Secondly, building upon insights gained from this case study, we
propose a set of data curation principles tailored for
government-released open data. These principles are designed to
address the unique challenges and requirements observed in the
curation of such datasets.


The remainder of this paper is organized as
follows. Section~\ref{sec:data} offers an in-depth review of the
history and current status of the 311 requests data. In
Section~\ref{sec:issues}, we identify and list potential issues
impacting data quality and curation
efficiency. Section~\ref{sec:improve} is dedicated to providing
actionable suggestions for mitigating or resolving these identified
issues. Following this, Section~\ref{sec:protocol} outlines a series
of general principles for the release of open data, drawing from our
findings. The paper concludes with a discussion in
Section~\ref{sec:disc}, encapsulating the key insights and
implications of our research.


\section{NYC 311 Requests Data} \label{sec:data}

The NYC 311 service, a critical component of New York City's public
engagement and service response framework, serves as a centralized hub
for non-emergency inquiries and requests. Introduced in 2010, the NYC
311 system was designed to streamline the city's response to
non-emergency issues, ranging from noise complaints to street
maintenance requests. The evolution of this system can be traced from
its initial implementation as a simple inquiry channel to a
comprehensive data management system that handles millions of requests
annually. Key milestones in its development include [list any
significant changes or technological upgrades].


Today, the NYC 311 data system is a robust platform that manages a
vast array of urban living-related inquiries. As of [current year or
most recent data year], the system handles approximately [number]
requests per year, covering categories like [list a few
categories]. The data, managed through [describe the current data
management system or software], is a valuable resource for city
administration and policy-making. It is publicly accessible through
[provide details of the data access platform, e.g., an online
portal]. This open data initiative enables not only governmental
transparency but also empowers researchers, civic developers, and the
general public. Examples of its use include [cite specific projects or
research that used this data].


Despite its success, the system faces challenges such as [list
challenges like data quality, privacy issues, or technological
constraints]. For instance, issues with [specific challenge] have
implications for [discuss the implications briefly].
The impact of the NYC 311 data extends beyond operational efficiency;
it has become instrumental in shaping city governance and community
engagement. The data has been pivotal in [give examples of policy
changes, community impacts, or improvements in city services].
The NYC 311 service exemplifies the dynamic nature of urban data
management and its critical role in modern governance. It provides an
invaluable resource for data scientists to practice data science
activities, including data curation.


We extracted [give the data period; extraction time] for this case
study. [justify the choices.]

\section{Potential Issues} \label{sec:issues}

Invalid data: closed data earlier than creation date;
latitude/longitude not in NYC; invalid zip code;


Inconsistent data: upper/lower cases; extra spaces; 
open text entries: provide enough major categories and sub categories;


Inefficiently stored data: lat/long versus (lat, long);
intersection street 1 versus cross street 1;
borough could be inferred from zipcode;
full name of agency not needed

\section{Improvements} \label{sec:improve}

Data validation: identify invalid data


Data cleaning: convert to all lower/upper cases;


Data organization: put zipcode/borough in a separate table; verify intersection
== cross; remove redundant columns (and compare how much space saved)


\section{Protocol Suggestions} \label{sec:protocol}



\section{Discussion} \label{sec:disc}


\bibliographystyle{asa}
\bibliography{ref}

\end{document}
